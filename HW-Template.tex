\documentclass{article}

\usepackage{fancyhdr}
\usepackage{extramarks}
\usepackage{amsmath}
\usepackage{amsthm}
\usepackage{amsfonts}
\usepackage{tikz}
\usepackage[plain]{algorithm}
\usepackage{algpseudocode}
\usepackage{enumerate}
\usepackage{enumitem}
\usepackage{listings}
\usepackage[numbered,framed]{matlab-prettifier}
\usepackage{hyperref}

\usetikzlibrary{automata,positioning}

%
% Basic Document Settings
%

\topmargin=-0.45in
\evensidemargin=0in
\oddsidemargin=0in
\textwidth=6.5in
\textheight=9.0in
\headsep=0.25in

\linespread{1.1}

\pagestyle{fancy}
\lhead{\hmwkAuthorName}
\rhead{\hmwkClass\ (\hmwkClassInstructor): \hmwkTitle}
%\rhead{\firstxmark}
%\lfoot{\lastxmark}
\cfoot{\thepage}

\renewcommand\headrulewidth{0.4pt}
\renewcommand\footrulewidth{0.4pt}

\setlength\parindent{0pt}

%
% Create Problem Sections
%

%\newcommand{\enterProblemHeader}[1]{
%    \nobreak\extramarks{}{Problem \arabic{#1} continued on next page\ldots}\nobreak{}
%    \nobreak\extramarks{Problem \arabic{#1} (continued)}{Problem \arabic{#1} continued on next page\ldots}\nobreak{}
%}
%
%\newcommand{\exitProblemHeader}[1]{
%    \nobreak\extramarks{Problem \arabic{#1} (continued)}{Problem \arabic{#1} continued on next page\ldots}\nobreak{}
%    \stepcounter{#1}
%    \nobreak\extramarks{Problem \arabic{#1}}{}\nobreak{}
%}

%\setcounter{secnumdepth}{0}
%\newcounter{partCounter}
%\newcounter{homeworkProblemCounter}
%\setcounter{homeworkProblemCounter}{1}
%\nobreak\extramarks{Problem \arabic{homeworkProblemCounter}}{}\nobreak{}

%
% Homework Problem Environment
%
% This environment takes an optional argument. When given, it will adjust the
% problem counter. This is useful for when the problems given for your
% assignment aren't sequential. See the last 3 problems of this template for an
% example.
%
%\newenvironment{homeworkProblem}[1][-1]{
%    \ifnum#1>0
%        \setcounter{homeworkProblemCounter}{#1}
%    \fi
%    \section{Problem \arabic{homeworkProblemCounter}}
%    \setcounter{partCounter}{1}
%    \enterProblemHeader{homeworkProblemCounter}
%}{
%    \exitProblemHeader{homeworkProblemCounter}
%}



%
% Homework Details
%   - Title
%   - Due date
%   - Class
%   - Section/Time
%   - Instructor
%   - Author
%

\newcommand{\hmwkTitle}{Homework  Fill Number} % Be sure to update this as needed!
\newcommand{\hmwkDueDate}{Fill Date} % Be sure to update this as needed!
\newcommand{\hmwkClass}{EC 415}
\newcommand{\hmwkClassTime}{}
\newcommand{\hmwkClassInstructor}{Professor David Starobinski}
\newcommand{\hmwkAuthorName}{Your Name} % Be sure to update this!
\newcommand{\hmwkKerberosName}{yourusername@bu.edu} % Be sure to update this!
%
% Title Page
%

\title{
    \vspace{2in}
    \textmd{\textbf{\hmwkClass:\ \hmwkTitle}}\\
    \normalsize\vspace{0.1in}\small{Due\ by\ \hmwkDueDate\ }\\
    \vspace{0.1in}\large{\textit{\hmwkClassInstructor\ \hmwkClassTime}}
    \vspace{3in}
}

\author{\textbf{\hmwkAuthorName} \\* \hmwkKerberosName}
\date{}

%\renewcommand{\part}[1]{\textbf{\large Part \Alph{partCounter}}\stepcounter{partCounter}\\}

%
% Various Helper Commands
%

% Useful for algorithms
\newcommand{\alg}[1]{\textsc{\bfseries \footnotesize #1}}

% For derivatives
\newcommand{\deriv}[1]{\frac{\mathrm{d}}{\mathrm{d}x} (#1)}

% For partial derivatives
\newcommand{\pderiv}[2]{\frac{\partial}{\partial #1} (#2)}

% Integral dx
\newcommand{\dx}{\mathrm{d}x}

% Alias for the Solution section header
\newcommand{\solution}{\textbf{\large Solution}}

% Probability commands: Expectation, Variance, Covariance, Bias
\newcommand{\E}{\mathrm{E}}
\newcommand{\Var}{\mathrm{Var}}
\newcommand{\Cov}{\mathrm{Cov}}
\newcommand{\Bias}{\mathrm{Bias}}


%DO NOT REMOVE ANYTHING ABOVE THIS POINT. 

%This is the beggining of this LaTeX document that serves as Homework Template for EC415, Spring 2019. The template was created by Natasa Trkulja (ntrkulja@bu.edu) and Prof. David Starobinski (staro@bu.edu). 

\begin{document}

\maketitle

\pagebreak

%% Listing the question is optional
%% Providing the solution is required

\section*{Problem 1}

What is Wi-Fi? What frequencies does it operate on? List out various 802.11 protocols and their respective frequencies and data rates. Finally, specify three factors that contribute to the weakening of Wi-Fi signals as they propagate in space. \\

\solution

\textbf{Wi-Fi} is technology for radio wireless local area networking of devices based on the IEEE 802.11 standards. \textit{Wi-Fi} is a trademark of the Wi-Fi Alliance, which restricts the use of the term \textit{Wi-Fi Certified} to products that successfully complete interoperability certification testing.\footnote{ "What is Wi-Fi (IEEE 802.11x)? A Webopedia Definition", \url{https://www.webopedia.com/TERM/W/Wi_Fi.html}} \\

Wi-Fi operates on the following frequencies:
\begin{itemize} % You can use \begin{itemize} to list items.
\item 2.4 GHz
\item 5 GHz
\item 60 GHz
\end{itemize}

The table below lists out different Wi-Fi protocols, the frequency on which they operate and data rates associated with them.\footnote{"Downloading the newest Wi-Fi protocols: 802.11ax and 802.11ay explained", \url{https://arstechnica.com/gadgets/2018/11/802-eleventy-which-802-11ax-and-802-11ay-explained/}}

\begin{center}
\begin{tabular}{ |c|c|c| } % \begin{tabular} is used for tables. Each |c| stands for a single column. |c|c|c| creates a table with 3 columns.
 \hline % hline creates a lines between two rows.
 \textbf{Designation} & \textbf{Spectrum} & \textbf{single-MIMO PHY} \\
 \hline
 802.11a & 5 GHz & 54 Mbps \\
 802.11b & 2.4 GHz & 11 Mbps \\	
 802.11g & 2.4 GHz & 54 Mbps \\	
 802.11n & 2.4 GHz / 5 GHz & 144 Mbps / 300 Mbps \\
 802.11ac & 5 GHz & 433 Mbps \\
 802.11ad & 60 GHz & ~5 Gbps \\	
 802.11ax (draft) & 2.4 GHz / 5 GHz & ~500 Mbps	\\
 802.11ay (draft) & 60 GHz & ~ 40 Gbps \\
 \hline
\end{tabular}
\end{center}

The three concepts that explain weakening of the Wi-Fi signal strength as it propagates in space are:
\begin{enumerate} % \begin{enumerate} can be used for numbered lists
\item Path loss
\item Shadowing
\item Multipath fading
\end{enumerate}


\section*{Problem 2}

Find the Fourier Transform of the rectangular pulse: 

% \begin{equation} is used for typing equations.
\begin{equation} 
\Pi(t) =
	\begin{cases}
      1, & -T/2 \leq t \leq T/2 \\
      0, & \text{otherwise}
    \end{cases}
\end{equation} 

\pagebreak % Use this when and where needed.

\solution 

The Fourier Transform of the rectangular pulse is calculated in the following manner:

\begin{equation}
\begin{split}
W(f) = \int_{t = -\infty}^{\infty} \Pi (t)e^{-j2\pi ft} \:dt = \int_{t = -T/2}^{T/2} (1)e^{-j2\pi ft} \:dt = \frac{e^{-j2\pi ft}}{-j2\pi f}\bigg\vert_{t = -T/2}^{T/2} \\  
= \frac{e^{-j\pi fT} - e^{j\pi fT}}{-j2\pi f} = T\frac{\sin(\pi fT)}{\pi fT} \equiv T\mbox{sinc} (fT)
\end{split}
\end{equation}

% Note that \: creates a space.

\section*{Problem 3}

Write a Matlab code to upconvert an input sine signal (frequency $f_i$) to a higher frequency ($f_c$) of a carrier so that it can more easily propagate over long distances and plot the result. Be sure to also plot the input and the carrier signal. \\

\solution 

Here is the Matlab code: 

\lstinputlisting[caption = {Sample code from Matlab}, language = Matlab]{modulate.m} % \lstinputlisting allows you to input Matlab code from a file.

The resulting plots are shown in the figures below.

% This is how you insert figures. \begin{figure}[H] is going to put the figure at the exact location of the Latex code, but you can also choose to put it on the top of the page by using \begin{figure}[t], bottom by using  \begin{figure}[b], etc. See for more: https://www.overleaf.com/learn/latex/Inserting_Images
\begin{figure}[H]
\centering
\includegraphics[width=0.5\textwidth]{carrier} % {...} is the name of your file, no extension needed. I named my image carrier in this instance and used .png extension, but .jpg also works.
\caption{Carrier signal}
\end{figure}

\begin{figure}[H]
\centering
\includegraphics[width=0.5\textwidth]{input}
\caption{Input signal}
\end{figure}

\begin{figure}[H]
\centering
\includegraphics[width=0.5\textwidth]{output}
\caption{Output signal}
\end{figure}
 

\vspace{5mm} %5mm vertical space, use as needed.



\end{document}