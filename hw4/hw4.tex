%%%%%%%%%%%%%%%%%%%%%%%%%%%%%%%%%%%%%%%%%%%%%%%%%%%%%%%%%%%%%%%
% This is the beggining of this LaTeX document that serves as 
% Homework Template for EC415, Spring 2019. The template was 
% created by Natasa Trkulja (ntrkulja@bu.edu) and 
% Prof. David Starobinski (staro@bu.edu). 
%%%%%%%%%%%%%%%%%%%%%%%%%%%%%%%%%%%%%%%%%%%%%%%%%%%%%%%%%%%%%%%
\documentclass{article}
% PACKAGES
\usepackage{fancyhdr}
\usepackage{extramarks}
\usepackage{amsmath}
\usepackage{amsthm}
\usepackage{amsfonts}
\usepackage{tikz}
\usepackage[plain]{algorithm}
\usepackage{algpseudocode}
\usepackage{enumerate}
\usepackage{enumitem}
\usepackage{listings}
\usepackage[numbered,framed]{matlab-prettifier}
\usepackage{hyperref}
\usepackage{subcaption}

\usetikzlibrary{automata,positioning}

%
% Basic Document Settings
%

\topmargin=-0.45in
\evensidemargin=0in
\oddsidemargin=0in
\textwidth=6.5in
\textheight=9.0in
\headsep=0.25in

\linespread{1.1}

\pagestyle{fancy}
\lhead{\hmwkAuthorName}
\rhead{\hmwkClass\ (\hmwkClassInstructor): \hmwkTitle}
%\rhead{\firstxmark}
%\lfoot{\lastxmark}
\cfoot{\thepage}

\renewcommand\headrulewidth{0.4pt}
\renewcommand\footrulewidth{0.4pt}

\setlength\parindent{0pt}

%
% Create Problem Sections
%

%\newcommand{\enterProblemHeader}[1]{
%    \nobreak\extramarks{}{Problem \arabic{#1} continued on next page\ldots}\nobreak{}
%    \nobreak\extramarks{Problem \arabic{#1} (continued)}{Problem \arabic{#1} continued on next page\ldots}\nobreak{}
%}
%
%\newcommand{\exitProblemHeader}[1]{
%    \nobreak\extramarks{Problem \arabic{#1} (continued)}{Problem \arabic{#1} continued on next page\ldots}\nobreak{}
%    \stepcounter{#1}
%    \nobreak\extramarks{Problem \arabic{#1}}{}\nobreak{}
%}

%\setcounter{secnumdepth}{0}
%\newcounter{partCounter}
%\newcounter{homeworkProblemCounter}
%\setcounter{homeworkProblemCounter}{1}
%\nobreak\extramarks{Problem \arabic{homeworkProblemCounter}}{}\nobreak{}

\newcommand{\hmwkTitle}{Homework  4} % Be sure to update this as needed!
\newcommand{\hmwkDueDate}{Friday 04/09/2021 6:00PM} % Be sure to update this as needed!
\newcommand{\hmwkClass}{EC 415}
\newcommand{\hmwkClassTime}{}
\newcommand{\hmwkClassInstructor}{Professor David Starobinski}
\newcommand{\hmwkAuthorName}{Michael Kremer} % Be sure to update this!
\newcommand{\hmwkKerberosName}{kremerme@bu.edu} % Be sure to update this!
%
% Title Page
%

\title{
    \vspace{2in}
    \textmd{\textbf{\hmwkClass:\ \hmwkTitle}}\\
    \normalsize\vspace{0.1in}\small{Due\ by\ \hmwkDueDate\ }\\
    \vspace{0.1in}\large{\textit{\hmwkClassInstructor\ \hmwkClassTime}}
    \vspace{3in}
}

\author{\textbf{\hmwkAuthorName} \\* \hmwkKerberosName}
\date{}

%\renewcommand{\part}[1]{\textbf{\large Part \Alph{partCounter}}\stepcounter{partCounter}\\}

%%%%%%%%%%%%%%%%%%%%%%%%%%%%%%%%%%%%%%%%%%%%%%%%%%%%%%%%%%%%%%%
% Various Helper Commands
%%%%%%%%%%%%%%%%%%%%%%%%%%%%%%%%%%%%%%%%%%%%%%%%%%%%%%%%%%%%%%%
\newcommand{\alg}[1]{\textsc{\bfseries \footnotesize #1}}   % Useful for algorithms
\newcommand{\deriv}[1]{\frac{\mathrm{d}}{\mathrm{d}x} (#1)} % For derivatives
\newcommand{\pderiv}[2]{\frac{\partial}{\partial #1} (#2)}  % For partial derivatives
\newcommand{\dx}{\mathrm{d}x}                               % Integral dx
\newcommand{\solution}{\textbf{\large Solution}}            % Alias for the Solution section header
\newcommand{\E}{\mathrm{E}}                                 % Probability commands: Expectation, 
\newcommand{\Var}{\mathrm{Var}}                             %                       Variance,
\newcommand{\Cov}{\mathrm{Cov}}                             %                       Covariance,
\newcommand{\Bias}{\mathrm{Bias}}                           %                       Bias
%%%%%%%%%%%%%%%%%%%%%%%%%%%%%%%%%%%%%%%%%%%%%%%%%%%%%%%%%%%%%%%
% DO NOT REMOVE ANYTHING ABOVE THIS POINT. 
%%%%%%%%%%%%%%%%%%%%%%%%%%%%%%%%%%%%%%%%%%%%%%%%%%%%%%%%%%%%%%%


%%%%%%%%%%%%%%%%%%%%%%%%%%%%%%%%%%%%%%%%%%%%%%%%%%%%%%%%%%%%%%%
% BEGIN & TITLE PAGE
%%%%%%%%%%%%%%%%%%%%%%%%%%%%%%%%%%%%%%%%%%%%%%%%%%%%%%%%%%%%%%%
\begin{document}
\maketitle
\pagebreak

%%%%%%%%%%%%%%%%%%%%%%%%%%%%%%%%%%%%%%%%%%%%%%%%%%%%%%%%%%%%%%%
% PROBLEM 4.21 (page 78)
%%%%%%%%%%%%%%%%%%%%%%%%%%%%%%%%%%%%%%%%%%%%%%%%%%%%%%%%%%%%%%%
\section*{Exercise 4.21}

Suppose that the noise in improvesnr.m is replaced with narrowband noise (as discussed in Section 4.1.3). Investigate the improvements in SNR
\begin{enumerate}[label=\alph*.]
    \item when the narrowband interference occurs outside the 3000 to 4000 Hz passband,
    \item when the narrowband interference occurs inside the 3000 to 4000 Hz passband.
\end{enumerate}
For part (a) use \(n=0.1*(\cos(2*pi*f1*t) + \cos(2*\pi*f_2*t))\) to model narrowband noise around the frequencies $f_1$ and $f_2$. Choose $f_1$ = 2000 Hz and $f_2$ = 5000Hz.
\\For part (b) use \(n=0.1*\cos(2*\pi*f3*t)\) to model narrowband noise around the frequency $f_3$. Choose $f_3$ = 3500 Hz.

%%%%%%%%%%%%%%%
% SOLUTION 4.21
%%%%%%%%%%%%%%%
\solution

\begin{enumerate}[label=\alph*.]
    \item TODO
    \item TODO
\end{enumerate}
\pagebreak

%%%%%%%%%%%%%%%%%%%%%%%%%%%%%%%%%%%%%%%%%%%%%%%%%%%%%%%%%%%%%%%
% PROBLEM 5.9 (page 88)
%%%%%%%%%%%%%%%%%%%%%%%%%%%%%%%%%%%%%%%%%%%%%%%%%%%%%%%%%%%%%%%
\section*{Exercise 5.9}

\begin{figure}[H]
	\centering
	\includegraphics[width=1\textwidth]{fig_5_8} 
	\caption{Figure 5.8}
\end{figure}
Consider the system shown in Figure 5.8. Show that the output of the system is \(2A_0w(t)\cos(2\pi f_ct)\), as indicated.

%%%%%%%%%%%%%%
% SOLUTION 5.9
%%%%%%%%%%%%%%
\solution

TODO
\pagebreak

%%%%%%%%%%%%%%%%%%%%%%%%%%%%%%%%%%%%%%%%%%%%%%%%%%%%%%%%%%%%%%%
% PROBLEM 5.12 (page 89)
%%%%%%%%%%%%%%%%%%%%%%%%%%%%%%%%%%%%%%%%%%%%%%%%%%%%%%%%%%%%%%%
\section*{Exercise 5.12}

\begin{figure}[H]
	\centering
	\includegraphics[width=1\textwidth]{fig_5_9} 
	\caption{Figure 5.9}
\end{figure}
Consider the transmission system of Figure 5.9. The message signal $w(t)$ has the magnitude spectrum shown in part (a). The transmitter in part (b) produces the transmitted signal x(t), which passes through the chan- nel in part (c). The channel scales the signal and adds narrowband interferers to create the received signal r(t). The transmitter and channel parameters are φ1 = 0.3 radians, f1 = 24.1 kHz, f2 = 23.9 kHz, f3 = 27.5 kHz, f4 = 29.3 kHz, and f5 = 22.6 kHz. The receiver processing r(t) is shown in Figure 5.9(d). All bandpass and lowpass filters are considered ideal, with a gain of unity in the passband and zero in the stopband.
\begin{enumerate}[label=\alph*.]
    \item Sketch \(|R(f)|\) for −30 kHz $\leq$ f $\leq$ 30 kHz. Clearly indicate the amplitudes and frequencies of key points in the sketch.
    \item Assume that $\phi_2$ is chosen to maximize the magnitude of $y(t)$ and reflects the value of $\phi_1$ and the delays imposed by the two ideal bandpass filters that form the received signal $r(t)$. Select the receiver parameters $f_6$, $f_7$, $f_8$, and $f_9$, so the receiver output $y(t)$ is a scaled version of $w(t)$.
\end{enumerate}

%%%%%%%%%%%%%%%
% SOLUTION 5.12
%%%%%%%%%%%%%%%
\solution

TODO
\pagebreak

%%%%%%%%%%%%%%%%%%%%%%%%%%%%%%%%%%%%%%%%%%%%%%%%%%%%%%%%%%%%%%%
% PROBLEM 5.16 (page 92)
%%%%%%%%%%%%%%%%%%%%%%%%%%%%%%%%%%%%%%%%%%%%%%%%%%%%%%%%%%%%%%%
\section*{Exercise 5.16}


\begin{figure}[H]
	\centering
	\includegraphics[width=1\textwidth]{fig_5_11} 
	\caption{Figure 5.11}
\end{figure}
Consider the scheme shown in Figure 5.11. The absolute band- width of the baseband signal $x_1$ is 6 MHz and that of the baseband signal $x_2(t)$ is 4MHz, $f_1$ = 164MHz, $f_2$ = 154MHz, $f_3$ = 148MHz, $f_4$ = 160MHz, $f_5$ = 80MHz, \(\phi=\pi/2\),and $f_6$ = 82MHz.
\begin{enumerate}[label=\alph*.]
    \item What is the absolute bandwidth of $x_3(t)$? 
    \item What is the absolute bandwidth of $x_5(t)$? 
    \item What is the absolute bandwidth of $x_6(t)$?
    \item What is the maximum frequency in $x_3(t)$? 
    \item What is the maximum frequency in $x_5(t)$?
\end{enumerate}

%%%%%%%%%%%%%%%
% SOLUTION 5.16
%%%%%%%%%%%%%%%
\solution

\begin{enumerate}[label=\alph*.]
    \item TODO 
    \item TODO 
    \item TODO
    \item TODO
    \item TODO
\end{enumerate}
\pagebreak

%%%%%%%%%%%%%%%%%%%%%%%%%%%%%%%%%%%%%%%%%%%%%%%%%%%%%%%%%%%%%%%
% QUESTION 5 
%%%%%%%%%%%%%%%%%%%%%%%%%%%%%%%%%%%%%%%%%%%%%%%%%%%%%%%%%%%%%%%
\section*{Question 5}

Consider the last line of AMlarge.m (see Listing 5.1):\\
\(envv=(\pi/2)*filter(b,1,abs(v))\);\\
Why is the output of the filter multiplied by the constant $\pi/2$? Justify your answer.

%%%%%%%%%%%%%
% SOLUTION Q5
%%%%%%%%%%%%%
\solution

TODO
\pagebreak

%%%%%%%%%%%%%%%%%%%%%%%%%%%%%%%%%%%%%%%%%%%%%%%%%%%%%%%%%%%%%%%
% QUESTION 6
%%%%%%%%%%%%%%%%%%%%%%%%%%%%%%%%%%%%%%%%%%%%%%%%%%%%%%%%%%%%%%%
\section*{Question 6}


The attached qam$\_$hw.mat file is a QAM passband signal 𝑣 that is the sum of two modulated messages $w_1$ and $w_2$.
These messages were respectively modulated using cosine and sine functions, with carrier signal f = 1000 Hz. The sampling period is T = 1/10000s and the total
duration of the signal is 0.3s. Note that these parameters are all the same as in the file AM.m (listing 5.2 in the textbook).
Plot the following:
\begin{enumerate}
    \item The modulated signal v.
    \item The demodulated signals (before the LPF) $x_1$ and $x_2$.
    \item The recovered signals (after the LPF) $m_1$ and $m_2$.
\end{enumerate}

Hints:
\begin{enumerate}
    \item To load the QAM signal 𝑣, use the command: load$('qam\_hw.mat','v')$;
    \item Use the same LPF parameters as in AM.m.
    \item The x-axis should be [0, 0.3] for all the plots.
    \item For the signal v, the y-axis should be [−11,11].
    \item For the signal $x_1$ and $m_1$, the y-axis should be [−5,10].
    \item For the signals $x_2$ and $m_2$, the y-axis should be [−10,1].
\end{enumerate}

%%%%%%%%%%%%%
% SOLUTION Q6
%%%%%%%%%%%%%
\solution

\begin{enumerate}
    \item TODO
    \item TODO
    \item TODO
\end{enumerate}
\pagebreak

%%%%%%%%%%%%%%%
% END
\end{document}

%%%%%%%%%%%%%%%%%%%%%%%%%%%%%%%%%%%%%%%%%%%%%%%%%%%%%%%%%%%%%%%
% USEFUL THINGS
%%%%%%%%%%%%%%%%%%%%%%%%%%%%%%%%%%%%%%%%%%%%%%%%%%%%%%%%%%%%%%%

%%%% INSERT MATLAB CODE %%%%
%\lstinputlisting[caption = {MATLAB code for Exercise 3.26b}, language = Matlab]{p3_26b.m}


%%%% INSERT IMAGE %%%%
%\begin{figure}[H]
%	\centering
%	\includegraphics[width=1\textwidth]{p3_9a} 
%	\caption{Passes all frequencies above 500 Hz}
%\end{figure}
