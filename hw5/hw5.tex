%%%%%%%%%%%%%%%%%%%%%%%%%%%%%%%%%%%%%%%%%%%%%%%%%%%%%%%%%%%%%%%
% This is the beggining of this LaTeX document that serves as 
% Homework Template for EC415, Spring 2019. The template was 
% created by Natasa Trkulja (ntrkulja@bu.edu) and 
% Prof. David Starobinski (staro@bu.edu). 
%%%%%%%%%%%%%%%%%%%%%%%%%%%%%%%%%%%%%%%%%%%%%%%%%%%%%%%%%%%%%%%
\documentclass{article}
% PACKAGES
\usepackage{fancyhdr}
\usepackage{extramarks}
\usepackage{amsmath}
\usepackage{amsthm}
\usepackage{amsfonts}
\usepackage{tikz}
\usepackage[plain]{algorithm}
\usepackage{algpseudocode}
\usepackage{enumerate}
\usepackage{enumitem}
\usepackage{listings}
\usepackage[numbered,framed]{matlab-prettifier}
\usepackage{hyperref}
\usepackage{subcaption}

\usetikzlibrary{automata,positioning}

%
% Basic Document Settings
%

\topmargin=-0.45in
\evensidemargin=0in
\oddsidemargin=0in
\textwidth=6.5in
\textheight=9.0in
\headsep=0.25in

\linespread{1.1}

\pagestyle{fancy}
\lhead{\hmwkAuthorName}
\rhead{\hmwkClass\ (\hmwkClassInstructor): \hmwkTitle}
%\rhead{\firstxmark}
%\lfoot{\lastxmark}
\cfoot{\thepage}

\renewcommand\headrulewidth{0.4pt}
\renewcommand\footrulewidth{0.4pt}

\setlength\parindent{0pt}

%
% Create Problem Sections
%

%\newcommand{\enterProblemHeader}[1]{
%    \nobreak\extramarks{}{Problem \arabic{#1} continued on next page\ldots}\nobreak{}
%    \nobreak\extramarks{Problem \arabic{#1} (continued)}{Problem \arabic{#1} continued on next page\ldots}\nobreak{}
%}
%
%\newcommand{\exitProblemHeader}[1]{
%    \nobreak\extramarks{Problem \arabic{#1} (continued)}{Problem \arabic{#1} continued on next page\ldots}\nobreak{}
%    \stepcounter{#1}
%    \nobreak\extramarks{Problem \arabic{#1}}{}\nobreak{}
%}

%\setcounter{secnumdepth}{0}
%\newcounter{partCounter}
%\newcounter{homeworkProblemCounter}
%\setcounter{homeworkProblemCounter}{1}
%\nobreak\extramarks{Problem \arabic{homeworkProblemCounter}}{}\nobreak{}

\newcommand{\hmwkTitle}{Homework  5} % Be sure to update this as needed!
\newcommand{\hmwkDueDate}{Friday 04/23/2021 6:00PM} % Be sure to update this as needed!
\newcommand{\hmwkClass}{EC 415}
\newcommand{\hmwkClassTime}{}
\newcommand{\hmwkClassInstructor}{Professor David Starobinski}
\newcommand{\hmwkAuthorName}{Michael Kremer} % Be sure to update this!
\newcommand{\hmwkKerberosName}{kremerme@bu.edu} % Be sure to update this!
%
% Title Page
%

\title{
    \vspace{2in}
    \textmd{\textbf{\hmwkClass:\ \hmwkTitle}}\\
    \normalsize\vspace{0.1in}\small{Due\ by\ \hmwkDueDate\ }\\
    \vspace{0.1in}\large{\textit{\hmwkClassInstructor\ \hmwkClassTime}}
    \vspace{3in}
}

\author{\textbf{\hmwkAuthorName} \\* \hmwkKerberosName}
\date{}

%\renewcommand{\part}[1]{\textbf{\large Part \Alph{partCounter}}\stepcounter{partCounter}\\}

%%%%%%%%%%%%%%%%%%%%%%%%%%%%%%%%%%%%%%%%%%%%%%%%%%%%%%%%%%%%%%%
% Various Helper Commands
%%%%%%%%%%%%%%%%%%%%%%%%%%%%%%%%%%%%%%%%%%%%%%%%%%%%%%%%%%%%%%%
\newcommand{\alg}[1]{\textsc{\bfseries \footnotesize #1}}   % Useful for algorithms
\newcommand{\deriv}[1]{\frac{\mathrm{d}}{\mathrm{d}x} (#1)} % For derivatives
\newcommand{\pderiv}[2]{\frac{\partial}{\partial #1} (#2)}  % For partial derivatives
\newcommand{\dx}{\mathrm{d}x}                               % Integral dx
\newcommand{\solution}{\textbf{\large Solution}}            % Alias for the Solution section header
\newcommand{\E}{\mathrm{E}}                                 % Probability commands: Expectation, 
\newcommand{\Var}{\mathrm{Var}}                             %                       Variance,
\newcommand{\Cov}{\mathrm{Cov}}                             %                       Covariance,
\newcommand{\Bias}{\mathrm{Bias}}                           %                       Bias
%%%%%%%%%%%%%%%%%%%%%%%%%%%%%%%%%%%%%%%%%%%%%%%%%%%%%%%%%%%%%%%
% DO NOT REMOVE ANYTHING ABOVE THIS POINT. 
%%%%%%%%%%%%%%%%%%%%%%%%%%%%%%%%%%%%%%%%%%%%%%%%%%%%%%%%%%%%%%%


%%%%%%%%%%%%%%%%%%%%%%%%%%%%%%%%%%%%%%%%%%%%%%%%%%%%%%%%%%%%%%%
% BEGIN & TITLE PAGE
%%%%%%%%%%%%%%%%%%%%%%%%%%%%%%%%%%%%%%%%%%%%%%%%%%%%%%%%%%%%%%%
\begin{document}
\maketitle
\pagebreak

%%%%%%%%%%%%%%%%%%%%%%%%%%%%%%%%%%%%%%%%%%%%%%%%%%%%%%%%%%%%%%%
% PROBLEM 8.1 (page 154)
%%%%%%%%%%%%%%%%%%%%%%%%%%%%%%%%%%%%%%%%%%%%%%%%%%%%%%%%%%%%%%%
\section*{Exercise 8.1}

The Matlab code in naivecode.m, which is on the website, implements the translation from binary to 4-PAM (and back again) suggested in (8.2). 
Examine the resiliency of this translation to noise by plotting the number of errors as a function of the noise variance v. 
What is the largest variance for which no errors occur? 
At what variance are the errors near 50\%?\\

%%%%%%%%%%%%%%
% SOLUTION 8.1
%%%%%%%%%%%%%%
\solution

TODO
\pagebreak

%%%%%%%%%%%%%%%%%%%%%%%%%%%%%%%%%%%%%%%%%%%%%%%%%%%%%%%%%%%%%%%
% PROBLEM 8.2 (page 154)
% (call your file graycode.m and upload it on Blackboard)
%%%%%%%%%%%%%%%%%%%%%%%%%%%%%%%%%%%%%%%%%%%%%%%%%%%%%%%%%%%%%%%
\section*{Exercise 8.2}

A Gray code has the property that the binary representation for each symbol differs from its neighbors by exactly one bit.
A Gray code for the translation of binary into 4-PAM is\\
01 $\rightarrow$ +3\\ 
11 $\rightarrow$ +1\\
10 $\rightarrow$ −1\\
00 $\rightarrow$ −3\\
Mimic the code in naivecode.m to implement this alternative and plot the number of errors as a function of the noise variance v. Compare your answer with Exercise 8.1. Which code is better?\\

%%%%%%%%%%%%%%
% SOLUTION 8.2
%%%%%%%%%%%%%%
\solution

TODO
\pagebreak

%%%%%%%%%%%%%%%%%%%%%%%%%%%%%%%%%%%%%%%%%%%%%%%%%%%%%%%%%%%%%%%
% PROBLEM 8.5 (page 157)
% (call your file betterpulseshape.m and upload it on Blackboard)
%%%%%%%%%%%%%%%%%%%%%%%%%%%%%%%%%%%%%%%%%%%%%%%%%%%%%%%%%%%%%%%
\section*{Exercise 8.5}

Can you think of a pulse shape that will have a narrower bandwidth than either of the above but that will still be time limited by T ? 
Implement it by changing the definition of ps, and check to see whether you are correct.\\

%%%%%%%%%%%%%%
% SOLUTION 8.5
%%%%%%%%%%%%%%
\solution

TODO
\pagebreak

%%%%%%%%%%%%%%%%%%%%%%%%%%%%%%%%%%%%%%%%%%%%%%%%%%%%%%%%%%%%%%%
% PROBLEM 8.8 (page 160)
% (run the code 5 times for each case) 
% [You will be able to answer this question after the lecture of 04/20/2021]
%%%%%%%%%%%%%%%%%%%%%%%%%%%%%%%%%%%%%%%%%%%%%%%%%%%%%%%%%%%%%%%
\section*{Exercise 8.8}

Rerun correx.m with different amounts of noise. Try sd=0, 0.1, 0.3, 0.5, 1, 2. 
How large can the noise be made if the correlation is still to find the true location of the header?\\

%%%%%%%%%%%%%%%
% SOLUTION 8.8
%%%%%%%%%%%%%%%
\solution

TODO
\pagebreak

%%%%%%%%%%%%%%%%%%%%%%%%%%%%%%%%%%%%%%%%%%%%%%%%%%%%%%%%%%%%%%%
% QUESTION 5
%%%%%%%%%%%%%%%%%%%%%%%%%%%%%%%%%%%%%%%%%%%%%%%%%%%%%%%%%%%%%%%
\section*{Extra Question}

TODO\\
%%%%%%%%%%%%%
% SOLUTION Q5
%%%%%%%%%%%%%
\solution

TODO
\pagebreak

%%%%%%%%%%%%%%%
% END
\end{document}

%%%%%%%%%%%%%%%%%%%%%%%%%%%%%%%%%%%%%%%%%%%%%%%%%%%%%%%%%%%%%%%
% USEFUL THINGS
%%%%%%%%%%%%%%%%%%%%%%%%%%%%%%%%%%%%%%%%%%%%%%%%%%%%%%%%%%%%%%%

%%%% INSERT MATLAB CODE %%%%
%\lstinputlisting[caption = {MATLAB code for Exercise 3.26b}, language = Matlab]{p3_26b.m}


%%%% INSERT IMAGE %%%%
%\begin{figure}[H]
%	\centering
%	\includegraphics[width=1\textwidth]{p3_9a} 
%	\caption{Passes all frequencies above 500 Hz}
%\end{figure}
