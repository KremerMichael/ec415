%%%%%%%%%%%%%%%%%%%%%%%%%%%%%%%%%%%%%%%%%%%%%%%%%%%%%%%%%%%%%%%
% This is the beggining of this LaTeX document that serves as 
% Homework Template for EC415, Spring 2019. The template was 
% created by Natasa Trkulja (ntrkulja@bu.edu) and 
% Prof. David Starobinski (staro@bu.edu). 
%%%%%%%%%%%%%%%%%%%%%%%%%%%%%%%%%%%%%%%%%%%%%%%%%%%%%%%%%%%%%%%
\documentclass{article}
% PACKAGES
\usepackage{fancyhdr}
\usepackage{extramarks}
\usepackage{amsmath}
\usepackage{amsthm}
\usepackage{amsfonts}
\usepackage{tikz}
\usepackage[plain]{algorithm}
\usepackage{algpseudocode}
\usepackage{enumerate}
\usepackage{enumitem}
\usepackage{listings}
\usepackage[numbered,framed]{matlab-prettifier}
\usepackage{hyperref}
\usepackage{subcaption}

\usetikzlibrary{automata,positioning}

%
% Basic Document Settings
%

\topmargin=-0.45in
\evensidemargin=0in
\oddsidemargin=0in
\textwidth=6.5in
\textheight=9.0in
\headsep=0.25in

\linespread{1.1}

\pagestyle{fancy}
\lhead{\hmwkAuthorName}
\rhead{\hmwkClass\ (\hmwkClassInstructor): \hmwkTitle}
%\rhead{\firstxmark}
%\lfoot{\lastxmark}
\cfoot{\thepage}

\renewcommand\headrulewidth{0.4pt}
\renewcommand\footrulewidth{0.4pt}

\setlength\parindent{0pt}

%
% Create Problem Sections
%

%\newcommand{\enterProblemHeader}[1]{
%    \nobreak\extramarks{}{Problem \arabic{#1} continued on next page\ldots}\nobreak{}
%    \nobreak\extramarks{Problem \arabic{#1} (continued)}{Problem \arabic{#1} continued on next page\ldots}\nobreak{}
%}
%
%\newcommand{\exitProblemHeader}[1]{
%    \nobreak\extramarks{Problem \arabic{#1} (continued)}{Problem \arabic{#1} continued on next page\ldots}\nobreak{}
%    \stepcounter{#1}
%    \nobreak\extramarks{Problem \arabic{#1}}{}\nobreak{}
%}

%\setcounter{secnumdepth}{0}
%\newcounter{partCounter}
%\newcounter{homeworkProblemCounter}
%\setcounter{homeworkProblemCounter}{1}
%\nobreak\extramarks{Problem \arabic{homeworkProblemCounter}}{}\nobreak{}

\newcommand{\hmwkTitle}{Homework  2} % Be sure to update this as needed!
\newcommand{\hmwkDueDate}{Friday 03/05/2021 6:00PM} % Be sure to update this as needed!
\newcommand{\hmwkClass}{EC 415}
\newcommand{\hmwkClassTime}{}
\newcommand{\hmwkClassInstructor}{Professor David Starobinski}
\newcommand{\hmwkAuthorName}{Michael Kremer} % Be sure to update this!
\newcommand{\hmwkKerberosName}{kremerme@bu.edu} % Be sure to update this!
%
% Title Page
%

\title{
    \vspace{2in}
    \textmd{\textbf{\hmwkClass:\ \hmwkTitle}}\\
    \normalsize\vspace{0.1in}\small{Due\ by\ \hmwkDueDate\ }\\
    \vspace{0.1in}\large{\textit{\hmwkClassInstructor\ \hmwkClassTime}}
    \vspace{3in}
}

\author{\textbf{\hmwkAuthorName} \\* \hmwkKerberosName}
\date{}

%\renewcommand{\part}[1]{\textbf{\large Part \Alph{partCounter}}\stepcounter{partCounter}\\}

%%%%%%%%%%%%%%%%%%%%%%%%%%%%%%%%%%%%%%%%%%%%%%%%%%%%%%%%%%%%%%%
% Various Helper Commands
%%%%%%%%%%%%%%%%%%%%%%%%%%%%%%%%%%%%%%%%%%%%%%%%%%%%%%%%%%%%%%%
\newcommand{\alg}[1]{\textsc{\bfseries \footnotesize #1}}   % Useful for algorithms
\newcommand{\deriv}[1]{\frac{\mathrm{d}}{\mathrm{d}x} (#1)} % For derivatives
\newcommand{\pderiv}[2]{\frac{\partial}{\partial #1} (#2)}  % For partial derivatives
\newcommand{\dx}{\mathrm{d}x}                               % Integral dx
\newcommand{\solution}{\textbf{\large Solution}}            % Alias for the Solution section header
\newcommand{\E}{\mathrm{E}}                                 % Probability commands: Expectation, 
\newcommand{\Var}{\mathrm{Var}}                             %                       Variance,
\newcommand{\Cov}{\mathrm{Cov}}                             %                       Covariance,
\newcommand{\Bias}{\mathrm{Bias}}                           %                       Bias
%%%%%%%%%%%%%%%%%%%%%%%%%%%%%%%%%%%%%%%%%%%%%%%%%%%%%%%%%%%%%%%
% DO NOT REMOVE ANYTHING ABOVE THIS POINT. 
%%%%%%%%%%%%%%%%%%%%%%%%%%%%%%%%%%%%%%%%%%%%%%%%%%%%%%%%%%%%%%%


%%%%%%%%%%%%%%%%%%%%%%%%%%%%%%%%%%%%%%%%%%%%%%%%%%%%%%%%%%%%%%%
% BEGIN & TITLE PAGE
%%%%%%%%%%%%%%%%%%%%%%%%%%%%%%%%%%%%%%%%%%%%%%%%%%%%%%%%%%%%%%%
\begin{document}
\maketitle
\pagebreak

%%%%%%%%%%%%%%%%%%%%%%%%%%%%%%%%%%%%%%%%%%%%%%%%%%%%%%%%%%%%%%%
% PROBLEM 3.1
%%%%%%%%%%%%%%%%%%%%%%%%%%%%%%%%%%%%%%%%%%%%%%%%%%%%%%%%%%%%%%%
\section*{Exercise 3.1}

Use specsquare.m to investigate the relationship between the time behavior of the square wave and its spectrum. The Matlab command zoom on is often helpful for viewing details of the plots. 

\begin{enumerate}[label=\alph*.]
	\item Try square waves with different frequencies: f=20, 40, 100, 300 Hz. How do the time plots change? How do the spectra change?
	\item Try square waves of different lengths, time=1, 10, 100 seconds. How does the spectrum change in each case?
	\item Try different sampling times, $T_s$=1/100, 1/10000 seconds. How does the spectrum change in each case? \\
\end{enumerate} 

%%%%%%%%%%%%%%
% SOLUTION 3.1
%%%%%%%%%%%%%%
\solution

To obtain all solutions to this exercise, a copy of the provided specsquare.m was used. For each part of the exercise, the relevant variable was changed before re-running the script and generating a plot.

\lstinputlisting[caption = {MATLAB code for Exercise 3.1}, language = Matlab]{specsquare.m}

\begin{enumerate}[label=\alph*.]

	\item As the frequency increases, the density of square waves on the time plot increases. TODO how do the spectra change
	\begin{figure}[H]
		\centering
		\begin{subfigure}{.5\textwidth}
  			\centering
 			\includegraphics[width=.9\textwidth]{p3_1a20}
  			\caption{f=20Hz}
  			\label{fig:sub1}
		\end{subfigure}%
		\begin{subfigure}{.5\textwidth}
  			\centering
  			\includegraphics[width=.9\textwidth]{p3_1a40}
  			\caption{f=40Hz}
  			\label{fig:sub2}
		\end{subfigure}
		\label{fig:test}
	\end{figure}
	\begin{figure}[H]
		\centering
		\begin{subfigure}{.5\textwidth}
  			\centering
 			\includegraphics[width=.9\textwidth]{p3_1a100}
  			\caption{f=100Hz}
  			\label{fig:sub1}
		\end{subfigure}%
		\begin{subfigure}{.5\textwidth}
  			\centering
  			\includegraphics[width=.9\textwidth]{p3_1a300}
  			\caption{f=300Hz}
  			\label{fig:sub2}
		\end{subfigure}
		\caption{square waves of increasing frequency}
		\label{fig:test}
	\end{figure}

	\item As the length of time increases, the magnitude of the spectra increases.
	\begin{figure}[H]
		\centering
		\begin{subfigure}{.33\textwidth}
  			\centering
 			\includegraphics[width=.9\textwidth]{p3_1b1}
  			\caption{t=1sec}
  			\label{fig:sub1}
		\end{subfigure}%
		\begin{subfigure}{.33\textwidth}
  			\centering
  			\includegraphics[width=.9\textwidth]{p3_1b10}
  			\caption{t=10sec}
  			\label{fig:sub2}
		\end{subfigure}%
		\begin{subfigure}{.33\textwidth}
  			\centering
 			\includegraphics[width=.9\textwidth]{p3_1b100}
  			\caption{t=100sec}
 			\label{fig:sub3}
		\end{subfigure}
		\caption{longer running square waves.}
		\label{fig:test}
	\end{figure}

	\item As the sampling time decreases, the sampling frequency increases. This means the spectrum plot is a more complete picture with more plotted spectra. However, this does not change the actual shape of the spectrum.
	\begin{figure}[H]
		\centering
		\begin{subfigure}{.5\textwidth}
  			\centering
 			\includegraphics[width=.9\textwidth]{p3_1c100}
  			\caption{$T_s$=1/100}
  			\label{fig:sub1}
		\end{subfigure}%
		\begin{subfigure}{.5\textwidth}
  			\centering
  			\includegraphics[width=.9\textwidth]{p3_1c10000}
  			\caption{$T_s$=1/10000}
  			\label{fig:sub2}
		\end{subfigure}
		\caption{square waves of increasing sample rate}
		\label{fig:test}
	\end{figure}
\end{enumerate}
\pagebreak


%%%%%%%%%%%%%%%%%%%%%%%%%%%%%%%%%%%%%%%%%%%%%%%%%%%%%%%%%%%%%%%
% PROBLEM 3.3
%%%%%%%%%%%%%%%%%%%%%%%%%%%%%%%%%%%%%%%%%%%%%%%%%%%%%%%%%%%%%%%
\section*{Exercise 3.3}

Mimic the code in specsquare.m to find the spectrum of:
\begin{enumerate}[label=\alph*.]
	\item An exponential pulse \( s(t) = e^{-t} \) for \( 0 < t < 10\)
	\item A scaled exponential pulse \( s(t) = 5e^{-t} \) for \( 0 < t < 10\)
	\item A Gaussian pulse \( s(t) = e^{-t^2} \) for \( -2 < t < 2\)
	\item A Gaussian pulse \( s(t) = e^{-t^2} \) for \( -20 < t < 20\)
	\item The sinusoids \( s(t) = sin(2\pi ft + \phi )\) for \(f = 20, 100, 1000\) with \(\phi = 0, \pi /4, \pi /2\) and \( 0 < t < 10 \) \\
\end{enumerate}

%%%%%%%%%%%%%%
% SOLUTION 3.3
%%%%%%%%%%%%%%
\solution

\begin{enumerate}[label=\alph*.]
	\item Here is the MATLAB code:
	\lstinputlisting[caption = {MATLAB code for part a}, language = Matlab]{p3_3a.m} 
	The resulting plot is shown in the figure below.
	\begin{figure}[H]
		\centering
		\includegraphics[width=0.5\textwidth]{p3_3a} 
	\caption{p3.3a d}
	\end{figure}
	
	\item Here is the MATLAB code:
	\lstinputlisting[caption = {MATLAB code for part b}, language = Matlab]{p3_3b.m}
	The resulting plot is shown in the figure below.
	\begin{figure}[H]
		\centering
		\includegraphics[width=0.5\textwidth]{p3_3b} 
		\caption{p3.3b}
	\end{figure}
	
	\item Here is the MATLAB code:
	\lstinputlisting[caption = {MATLAB code for part c}, language = Matlab]{p3_3c.m}
	The resulting plot is shown in the figure below.
	\begin{figure}[H]
		\centering
		\includegraphics[width=0.5\textwidth]{p3_3c} 
		\caption{p3.3c}
	\end{figure}
	
	\item Here is the MATLAB code:
	\lstinputlisting[caption = {MATLAB code for part d}, language = Matlab]{p3_3d.m} 
	The resulting plot is shown in the figure below.
	\begin{figure}[H]
		\centering
		\includegraphics[width=0.5\textwidth]{p3_3d} 
		\caption{p3.3d}
	\end{figure}
	
	\item Here is the MATLAB code: TODO
\end{enumerate}
\pagebreak

%%%%%%%%%%%%%%%%%%%%%%%%%%%%%%%%%%%%%%%%%%%%%%%%%%%%%%%%%%%%%%%
% PROBLEM 3.6
%%%%%%%%%%%%%%%%%%%%%%%%%%%%%%%%%%%%%%%%%%%%%%%%%%%%%%%%%%%%%%%
\section*{Exercise 3.6}

Mimic the code in speccos.m to find the spectrum of a cosine wave
\begin{enumerate}[label=\alph*.]
	\item For different frequencies f=1, 2, 20, 30 Hz
	\item for different phases $\phi$ = 0, 0.1, $\pi$/8, $\pi$/2 radians
	\item For different sampling rates $T_s$=1/10, 1/1000, 1/100000.
\end{enumerate}

%%%%%%%%%%%%%%
% SOLUTION 3.6
%%%%%%%%%%%%%%
\solution

To obtain all solutions to this exercise, a copy of the provided speccos.m was used. For each part of the exercise, the relevant variable was changed before re-running the script and generating a plot.

\lstinputlisting[caption = {MATLAB code for Exercise 3.6}, language = Matlab]{speccos.m}

\begin{enumerate}[label=\alph*.]
	\item A
	\begin{figure}[H]
		\centering
		\begin{subfigure}{.5\textwidth}
  			\centering
 			\includegraphics[width=.9\textwidth]{p3_6a1}
  			\caption{f=1Hz}
  			\label{fig:sub1}
		\end{subfigure}%
		\begin{subfigure}{.5\textwidth}
  			\centering
  			\includegraphics[width=.9\textwidth]{p3_6a2}
  			\caption{f=2Hz}
  			\label{fig:sub2}
		\end{subfigure}
		\label{fig:test}
	\end{figure}
	\begin{figure}[H]
		\centering
		\begin{subfigure}{.5\textwidth}
  			\centering
 			\includegraphics[width=.9\textwidth]{p3_6a20}
  			\caption{f=20Hz}
  			\label{fig:sub1}
		\end{subfigure}%
		\begin{subfigure}{.5\textwidth}
  			\centering
  			\includegraphics[width=.9\textwidth]{p3_6a30}
  			\caption{f=30Hz}
  			\label{fig:sub2}
		\end{subfigure}
		\caption{square waves of increasing frequency}
		\label{fig:test}
	\end{figure}
	
	\item B
	\begin{figure}[H]
		\centering
		\begin{subfigure}{.5\textwidth}
  			\centering
 			\includegraphics[width=.9\textwidth]{p3_6b_1}
  			\caption{phi=0}
  			\label{fig:sub1}
		\end{subfigure}%
		\begin{subfigure}{.5\textwidth}
  			\centering
  			\includegraphics[width=.9\textwidth]{p3_6b0}
  			\caption{phi=2Hz}
  			\label{fig:sub2}
		\end{subfigure}
		\label{fig:test}
	\end{figure}
	\begin{figure}[H]
		\centering
		\begin{subfigure}{.5\textwidth}
  			\centering
 			\includegraphics[width=.9\textwidth]{p3_6b8}
  			\caption{$phi$=$pi$/8}
  			\label{fig:sub1}
		\end{subfigure}%
		\begin{subfigure}{.5\textwidth}
  			\centering
  			\includegraphics[width=.9\textwidth]{p3_6b2}
  			\caption{$phi$=$pi$/2}
  			\label{fig:sub2}
		\end{subfigure}
		\caption{square waves of increasing frequency}
		\label{fig:test}
	\end{figure}
	
	\item C
	\begin{figure}[H]
		\centering
		\begin{subfigure}{.33\textwidth}
  			\centering
 			\includegraphics[width=.9\textwidth]{p3_6c10}
  			\caption{$T_s$=1/10}
  			\label{fig:sub1}
		\end{subfigure}%
		\begin{subfigure}{.33\textwidth}
  			\centering
  			\includegraphics[width=.9\textwidth]{p3_6c1000}
  			\caption{$T_s$=1/1000}
  			\label{fig:sub2}
		\end{subfigure}%
		\begin{subfigure}{.33\textwidth}
  			\centering
 			\includegraphics[width=.9\textwidth]{p3_6c100000}
  			\caption{$T_s$=1/100000}
 			\label{fig:sub3}
		\end{subfigure}
		\caption{longer running square waves.}
		\label{fig:test}
	\end{figure}
\end{enumerate}
\pagebreak


%%%%%%%%%%%%%%%%%%%%%%%%%%%%%%%%%%%%%%%%%%%%%%%%%%%%%%%%%%%%%%%
% PROBLEM 3.9
%%%%%%%%%%%%%%%%%%%%%%%%%%%%%%%%%%%%%%%%%%%%%%%%%%%%%%%%%%%%%%%
\section*{Exercise 3.9}

Mimic the code in filternoise.m to create a filter that
\begin{enumerate}[label=\alph*.]
	\item Passes all frequencies above 500 Hz
	\item Passes all frequencies below 3000 Hz
	\item Rejects all frequencies between 1500 and 2500 Hz
\end{enumerate}

%%%%%%%%%%%%%%
% SOLUTION 3.9
%%%%%%%%%%%%%%
\solution

\begin{enumerate}[label=\alph*.]
	\item A
	\lstinputlisting[caption = {MATLAB code for Exercise 3.9a}, language = Matlab]{p3_9a.m}
	
	\item B
	\lstinputlisting[caption = {MATLAB code for Exercise 3.9b}, language = Matlab]{p3_9b.m}
	
	\item C
	\lstinputlisting[caption = {MATLAB code for Exercise 3.9c}, language = Matlab]{p3_9c.m}
\end{enumerate}
\pagebreak


%%%%%%%%%%%%%%%%%%%%%%%%%%%%%%%%%%%%%%%%%%%%%%%%%%%%%%%%%%%%%%%
% PROBLEM 3.10
%%%%%%%%%%%%%%%%%%%%%%%%%%%%%%%%%%%%%%%%%%%%%%%%%%%%%%%%%%%%%%%
\section*{Exercise 3.10}

Change the sampling rate to $T_s$=1/20000. Redesign the three filters from Exercise 3.9.

%%%%%%%%%%%%%%%
% SOLUTION 3.10
%%%%%%%%%%%%%%%
\solution

soln here
\pagebreak


%%%%%%%%%%%%%%%%%%%%%%%%%%%%%%%%%%%%%%%%%%%%%%%%%%%%%%%%%%%%%%%
% PROBLEM 3.11
%%%%%%%%%%%%%%%%%%%%%%%%%%%%%%%%%%%%%%%%%%%%%%%%%%%%%%%%%%%%%%%
\section*{Exercise 3.11}

Let $x_1(t)$ be a cosine wave of frequency f = 800, $x_2(t)$ be a cosine wave of frequency f = 2000, and $x_3(t)$ be a cosine wave of frequency f = 4500. Let \(x(t) = x_1(t) + 0.5 * x_2(t) + 2 * x_3(t)\). Use x(t) as input to each of the three filters in filternoise.m. Plot the spectra, and explain what you see.

%%%%%%%%%%%%%%%
% SOLUTION 3.11
%%%%%%%%%%%%%%%
\solution
\lstinputlisting[caption = {MATLAB code for Exercise 3.11}, language = Matlab]{p3_11.m}
TODO: change matlab script and get spectra and explain
\pagebreak

%%%%%%%%%%%%%%%%%%%%%%%%%%%%%%%%%%%%%%%%%%%%%%%%%%%%%%%%%%%%%%%
% PROBLEM 3.26
%%%%%%%%%%%%%%%%%%%%%%%%%%%%%%%%%%%%%%%%%%%%%%%%%%%%%%%%%%%%%%%
\section*{Exercise 3.26}

Mimic the code in modulate.m to find the spectrum of the output y(t) of a modulator block (with modulation frequency $f_c$ = 1000 Hz) when
\begin{enumerate}[label=\alph*.]
	\item The input is \(x(t) = cos(2*\pi*f_1*t) + cos(2*\pi*f_2*t)\) for $f_1$ = 100 and $f_2$ = 150 Hz
	\item The input is a square wave with fundamental f = 150 Hz
	\item The input is a noise signal with all energy below 300 Hz
\end{enumerate}

%%%%%%%%%%%%%%%
% SOLUTION 3.26
%%%%%%%%%%%%%%%
\solution

\begin{enumerate}[label=\alph*.]
	\item A
	
	\item B
	
	\item C
\end{enumerate}
\pagebreak

%%%%%%%%%%%%%%%
% END
\end{document}